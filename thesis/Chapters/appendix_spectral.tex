\chapter{Derivation of integrated spectral function\label{app:spectral}}
\thispagestyle{chapterBeginStyle}
We want to prove equation~\eqref{eq:integrated spectral function}, which gives
a recipe for the numerical calculation of the integrated spectral function for an arbitrary observable \(A\).
Our starting point is Definition~\ref{def:spectral function}
\begin{align}
        S(\omega) &=  \lim _{\varepsilon \to 0^+} \frac{1}{2 \pi} \int_{-\infty}^{\infty} \mathrm{d} t 
        \; e^{i \omega t-|t| \varepsilon}\hs{A(t)}{A} = \lim _{\varepsilon \to 0^+} \frac{1}{2 \pi}
        \int_{-\infty}^{\infty} \mathrm{d} t \; e^{i \omega t-|t| \varepsilon} \frac{1}{\dimension}
        \tr\left[\left(e^{iHt}Ae^{-iHt}\right)^{\dagger}A\right] \nonumber\\
        &=\lim _{\varepsilon \to 0^+} \frac{1}{2 \pi}
        \int_{-\infty}^{\infty} \mathrm{d} t \; e^{i \omega t-|t| \varepsilon} \frac{1}{\dimension}
        \tr\left[e^{iHt}\left(\sum_{m} \ketbra{m}{m}\right) 
        A\left(\sum_{n} \ketbra{n}{n}\right)e^{-iHt}A\right] \nonumber\\
        &= \frac{1}{\dimension} \frac{1}{2 \pi}\sum_{n,m} \lim _{\varepsilon \to 0^+} 
        \int_{-\infty}^{\infty} \mathrm{d} t \; e^{i \omega t-|t| \varepsilon}
        \tr\left[e^{i E_m t}\ketbra{m}{m} A\ketbra{n}{n}e^{-i E_n t}A\right] \nonumber\\
        &= \frac{1}{\dimension} \frac{1}{2 \pi}\sum_{n,m} A_{mn}  \lim _{\varepsilon \to 0^+} 
        \int_{-\infty}^{\infty} \mathrm{d} t \; e^{i \omega t-|t| \varepsilon}e^{i (E_m-E_n) t}
        \sum_{k} \underbrace{\braket{k}{m}}_{=\delta_{km}} \matrixel{n}{A}{k} \nonumber\\
        &= \frac{1}{\dimension} \frac{1}{2 \pi}\sum_{n,m} \abs{A_{mn}}^2  \underbrace{\lim _{\varepsilon 
        \to 0^+} \int_{-\infty}^{\infty} \mathrm{d} t \; e^{i \omega t-|t| 
        \varepsilon}e^{i (E_m-E_n) t}}_{\mathcal{I}}\label{eq:spectral function simplified}
\end{align}
Let us now deal with the integral \(\mathcal{I}\)
\begin{align}
    \mathcal{I} &= \lim _{\varepsilon \to 0^+} \int_{-\infty}^{\infty} \mathrm{d} t \; 
    e^{ i (E_m-E_n+\omega)t -|t| \varepsilon} = \lim _{\varepsilon \to 0^+} 
    \Bigg[\lim_{T_1\to -\infty}\int_{T_1}^{0} \mathrm{d} t \; e^{ i (E_m-E_n+\omega -i \varepsilon)t}\nonumber\\
    &+\lim_{T_2\to \infty}\int_{0}^{T_2} \mathrm{d} t \;  e^{ i (E_m-E_n+\omega + i\varepsilon)t}\Bigg]
    = \lim _{\varepsilon \to 0^+} \Bigg[\lim_{T_1\to -\infty} 
    \frac{1-e^{ i (E_m-E_n+\omega )T_1} e^{\varepsilon T_1}}{i (E_m-E_n+\omega -i \varepsilon)}\nonumber\\
     &+ \lim_{T_2\to \infty} \frac{e^{ i (E_m-E_n+\omega )T_2} e^{-\varepsilon T_2}-1}
    {i (E_m-E_n+\omega +i \varepsilon)}\Bigg]
    = \lim _{\varepsilon \to 0^+} \left[\frac{i}{E_n-E_m-\omega +i \varepsilon} + 
    \frac{i}{E_m-E_n+\omega +i \varepsilon} \right]\nonumber\\
     &= \lim _{\varepsilon \to 0^+}
    \frac{2\varepsilon}{(E_m-E_n+\omega -i \varepsilon)(E_m-E_n+\omega +i \varepsilon)}
    =\lim _{\varepsilon \to 0^+} \frac{2\varepsilon}{(E_m-E_n+\omega)^2 +\varepsilon^2}
\end{align}
The obtained result is a limit of the so-called Poisson kernel. This happens to be
a representations of Dirac delta in the form of a limit of a sequence of functions~\autocite{byron1992mathematics}
\begin{equation}
    \lim _{\varepsilon \to 0^+} \frac{1}{\pi} \frac{\varepsilon}{x^2 +\varepsilon^2} = \delta(x) 
\end{equation}
Thus we get
\begin{equation}
    \mathcal{I} = \lim _{\varepsilon \to 0^+} \frac{2\varepsilon}{(E_m-E_n+\omega)^2 +\varepsilon^2}
    = 2\pi \delta(E_m-E_n+\omega)
\end{equation}
Inserting this result into equation~\eqref{eq:spectral function simplified} we get
\begin{align}
    S(\omega) &= \frac{1}{\dimension} \frac{1}{2 \pi}\sum_{n,m} \abs{A_{mn}}^2  
    \lim _{\varepsilon \to 0^+} \int_{-\infty}^{\infty} \mathrm{d} t \; e^{i \omega t-|t| 
    \varepsilon}e^{i (E_m-E_n) t} = \frac{1}{\dimension}\sum_{n,m} \abs{A_{mn}}^2 \delta(E_m-E_n+\omega)
\end{align}
We are now ready to compute the integrated spectral function
\begin{align}
    I(\omega) &= \int_{-\omega}^{\omega} \mathrm{d}\omega^{\prime} S(\omega^{\prime}) =
    \frac{1}{\dimension}\sum_{n,m} \abs{A_{mn}}^2 \int_{-\omega}^{\omega} \mathrm{d}\omega^{\prime} 
    \delta(E_m-E_n+\omega^{\prime}) \nonumber \\ &= \frac{1}{\dimension}\sum_{n,m} \abs{A_{mn}}^2
    \theta(\omega + (E_m-E_n))\theta(\omega - (E_m-E_n)) \nonumber\\
    &= \frac{1}{\dimension}\sum_{n,m} \abs{A_{mn}}^2\theta(\omega - \abs{E_m-E_n})
\end{align}


% \begin{figure}[htbp]
%     \centering
%     \begin{tikzpicture}[scale=1.4]
%       \colorlet{col1}{blue!30}
%     %  \begin{scope}[smooth,draw=gray!20,y=0.3989422804cm]
%     %       \filldraw[fill=col1] plot[id=f1, domain=-1.5:1.5,samples=100] function {6*exp(-6*x*x)};
%     %       \draw[black] plot[id=f2,domain=-1.5:1.5,samples=100]
%     %           function {6*exp(-6*x*x)};
%     %   \end{scope}
%       \draw[->] (-3,0) -- (3,0) node [right] {\(\omega^{\prime}\)};
%       \draw[-] (0,0) -- (0,0) node [midway,yshift=-8pt] {\(0\)};
%       \draw[dashed] (-1.5,0) -- (-1.5,2.5) node [midway,yshift=-57pt] {\(-\omega\)};
%       \draw[dashed] (1.5,0) -- (1.5,2.5) node [midway,yshift=-57pt] {\(\omega\)};
%       \draw[-] (2.3,0) -- (2.3,0) node [midway, yshift=30pt] {\(\omega-x < 0\)};
%       \draw[-] (2.3,0) -- (2.3,0) node [midway, yshift=20pt] {\(\omega+x > 0\)};
%       \draw[-] (-2.3,0) -- (-2.3,0) node [midway, yshift=30pt] {\(\omega-x > 0\)};
%       \draw[-] (-2.3,0) -- (-2.3,0) node [midway, yshift=20pt] {\(\omega+x < 0\)};
%       \draw[-] (0,0) -- (0,0) node [midway, yshift=30pt] {\(\omega-x > 0\)};
%       \draw[-] (0,0) -- (0,0) node [midway, yshift=20pt] {\(\omega+x > 0\)};
%     %   \draw[<-] (-1.5,0) -- (1.5,0) node [right] {$x$};
%     %   \draw[<->] (-0.37,1) -- (0.37,1) node [midway, yshift=6pt] {$\gamma$};
%     \end{tikzpicture}  
%     \caption{Illustration of the equality \(\int_{-\omega}^{\omega} \mathrm{d}\omega^{\prime} 
%     \delta(x-\omega^{\prime}) = \theta(\omega + x)\theta(\omega - x) = \theta(\omega-\abs{x})\).}
%     \label{fig:dirac delta}
%   \end{figure}