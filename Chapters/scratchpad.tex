\chapter{scratchpad}
\thispagestyle{chapterBeginStyle}


\section{Meetings}

\textcolor{red}{
    27.10.2021 meeting:
    \begin{itemize}
        \item integrability of Heisenberg model for \(\alpha = 0.0 \) is Bethe ansatz?
        {
            \begin{itemize}
                \item bethe ansatz and existence of extensive number of IOMs
            \end{itemize}
        }
        \item sources for motivation and history in intro
        {
            \begin{itemize}
                \item From Chaos to Quantum Thermalization\ldots
                \item arXiv:2012.07849
            \end{itemize}
        }
        \item best way to introduce Heisenberg model?
        {\begin{itemize}
            \item follow Dirac as in Spalek book
        \end{itemize}}
        \item source for nonlocal operators stiffness vanishing in thermodynamic limit
        {\begin{itemize}
            \item just use Zotos1997
        \end{itemize}}
        \item is extrapolation with 1/L just finite size scaling?
        {\begin{itemize}
            \item yes
        \end{itemize}}
    \end{itemize}
}

\textcolor{blue}{
    03.11.2021 meeting:
    \begin{itemize}
        \item is this algorithm valid for any lattice model? or just 1-D
    \end{itemize}
}

\section{Other}

\noindent Quantity that is plotted in Figures~4--14:
\begin{itemize}
    \item 
    With extrapolation to thermodynamic limit:
    \begin{equation}
        R_l(\tau,\alpha) = \frac{\lambda_l(L\rightarrow \infty,\tau,\alpha)}
        {\lambda_l(L\rightarrow \infty,\tau \rightarrow \infty,\alpha=0)}
        \label{eq:R1 extrap}
    \end{equation}
    \item
    Without extrapolation to thermodynamic limit:
    \begin{equation}
        R^L_l(\tau,\alpha) = \frac{\lambda_l(L,\tau,\alpha)}
        {\lambda_l(L,\tau \rightarrow \infty,\alpha=0)}
        \label{eq:R1 no extrap}
    \end{equation}
\end{itemize}


