\chapter{Summary}
\thispagestyle{chapterBeginStyle}
The aim of this work was to investigate the properties of noncommuting (quasi)local integrals of
motion within the integrable spin-\(1/2\) XXZ model with periodic boundary conditions
on one-dimensional lattice. The case of commuting (Q)LIOMs has been studied extensively
and it is now known that in nearly in the presence of integrability breaking perturbation, 
decay of their autocorrelation functions is exponential and the characteristic relaxation
time scales quadratically with perturbation strength. On the other hand, the noncommuting
(Q)LIOMs are much less studied. Even though their existence is a well-known fact that was
formally proven,  little has been known about their dynamics under weak integrability-breaking
perturbation, which is often present in real-world systems.

Such a numerical study was realized in this thesis. To this end, a procedure allowing for 
systematic identification of all (Q)LIOMs present in a given model was described and implemented.
It was then applied to the spin-\(1/2\) chain with different values of anisotropy parameter,
where to reduce the computational cost, the support of operators was restricted to \(3\) lattice sites.
As a test case, a known commuting LIOM was identified, namely, the energy current operator.
Apart from that, two nontrivial noncommuting conserved  quantities were found, one local for
\(\Delta = 1.0\) and one quasilocal for \(\Delta=0.5\), in agreement with the existing analytical
results. To study the decay of these quantities, the formalism of integrated spectral
functions was introduced. With its help, the known results about exponential
decay of energy current were confirmed. However, the key finding presented in this thesis 
is a novel result about the relaxation of noncommuting (Q)LIOMs. Both of them were found
to decay in a Gaussian-like manner, with characteristic relaxation time scaling linearly
with perturbation strength. This anomalous scaling was explained by observing, that
it is not the breaking of integrability that is responsible for the decay, but shifting
away from high-symmetry values of anisotropy parameter. Degeneracy resulting from
symmetries is removed already in the first order of perturbation theory, hence the linear scaling.

The border between chaos and integrability in quantum many-body systems is still full
of open problems, with as fundamental things as the concept of quantum integrability itself
being a topic of debate. A highly sought-after result would perhaps be the quantum analogue
of the famous KAM theorem, giving a concrete answer to the question about onset of chaos
in perturbed integrable systems. However, in spite of numerous research, it seems that
physicists still have a long road ahead of them. As for the work presented in this thesis,
one of possible future directions is an extension of the used here approach to the
case of two-dimensional lattice models, where considerable difficulties of computational
nature await.