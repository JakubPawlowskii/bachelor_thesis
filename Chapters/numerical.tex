\chapter{Numerical results}
\thispagestyle{chapterBeginStyle}


We conducted preliminary studies for small values of L, without assuming translational invariance. Available resources
allowed us to make unrestricted search for \(L = 8,9,10,11,12\) in case of \(m = 3\) and \(L = 8,9,10,11\) in case of
\(m = 4\). Nevertheless, operators that maximized stiffness for given \(L\) and \(\Delta \) turned out to be translationally
invariant. Therefore, we restrict our considerations to translationally invariant operators only. This allowed us
to obtain numerical results for \(L\) up to \(14\) in case of \(m = 3\) and up to \(13\) in case of \(m = 4\).
To study the case of  \(L = 16\) we considered a subspace of \(\mathcal{A}_L^m\) spanned by basis operators
\(\overline{O}_{\underline{s}}\) that have nonzero coefficients in operator with largest stiffness for \(L = 14\). 
Then we diagonalized the resulting \(2\cross 2\) correlation matrix to obtain the stiffness.

