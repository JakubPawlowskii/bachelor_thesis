\chapter{Derivation of spin energy current\label{app:spin energy current derivation}}
\thispagestyle{chapterBeginStyle}

Equation~\eqref{eq:energy current defining equation} is
conceptually simple, yet quite tedious to solve due to the number of commutators present. Luckily, leveraging the commutator properties
to our advantage allow us to simplify the calculations. Let us begin with inserting the definition of \(h_{k,k+1}\) into 
equation~\eqref{eq:energy current defining equation}
\begin{align}
    \comm{h_{k,k+1}}{h_{r,r+1}} =& \; \comm{J_{x} \Sx_{k}\Sx_{k+1} + J_{x} \Sy_{k}\Sy_{k+1} + J_{z} \Sz_{k}\Sz_{k+1}}{J_{x} \Sx_{r}\Sx_{r+1} + J_{x} \Sy_{r}\Sy_{r+1} + J_{z} \Sz_{r}\Sz_{r+1}}\nonumber\\
    =& \; J_{x} J_{y}\comm{\Sx_{k}\Sx_{k+1}}{\Sy_{r}\Sy_{r+1}} + J_{x} J_{z}\comm{\Sx_{k}\Sx_{k+1}}{\Sz_{r}\Sz_{r+1}} + J_{y} J_{x}\comm{\Sy_{k}\Sy_{k+1}}{\Sx_{r}\Sx_{r+1}}\nonumber\\
    +& \; J_{y} J_{z}\comm{\Sy_{k}\Sy_{k+1}}{\Sz_{r}\Sz_{r+1}} + J_{z} J_{x}\comm{\Sz_{k}\Sz_{k+1}}{\Sx_{r}\Sx_{r+1}} + J_{z} J_{y}\comm{\Sz_{k}\Sz_{k+1}}{\Sy_{r}\Sy_{r+1}}  
\end{align}
By inspection, it becomes clear that out of the six terms present, only three will need to be directly evaluated as commutators of the form
\(\comm{A}{B}\) will differ from \(\comm{B}{A}\) by a sign and index change.
\begin{align}
    J_{x} J_{y}\comm{\Sx_{k}\Sx_{k+1}}{\Sy_{r}\Sy_{r+1}} =& \; J_x J_y \Big(\Sx_k\comm{\Sx_{k+1}}{\Sy_{r}\Sy_{r+1}} + \comm{\Sx_{k}}{\Sy_{r}\Sy_{r+1}} \Sx_{k+1} \Big) \nonumber \\
    =& \; J_x J_y \Big( \Sx_k \left( \Sy_r \comm{\Sx_{k+1}}{\Sy_{r+1}} + \comm{\Sx_{k+1}}{\Sy_{r}} \Sy_{r+1}\right)\nonumber\\
    +& \left( \Sy_r \comm{\Sx_{k}}{\Sy_{r+1}} + \comm{\Sx_{k}}{\Sy_{r}} \Sy_{r+1} \right)\Sx_{k+1} \Big)\nonumber\\
    =& \; i J_x J_y \Big( \delta_{k+1,r+1} \Sx_k \Sy_r \Sz_{k+1} + \delta_{k+1,r} \Sx_k \Sz_{k+1} \Sy_{r+1} \nonumber\\
    +&\; \delta_{k,r+1} \Sy_r \Sz_k \Sx_{k+1} +\delta_{k,r} \Sz_k \Sy_{r+1} \Sx_{k+1}\Big)
\end{align}
Carrying out the calculation of the remaining two nontrivial commutators, we arrive at the following equations
\begin{align*}
    J_{z} J_{x}\comm{\Sz_{k}\Sz_{k+1}}{\Sx_{r}\Sx_{r+1}} =&\; i J_z J_x \Big( \delta_{k+1,r+1} \Sx_r \Sz_k \Sy_{r+1} + \delta_{k+1,r} \Sz_k \Sy_{r} \Sx_{r+1} \\
    +&\; \delta_{k,r+1} \Sx_r \Sy_{r+1} \Sz_{k+1} +\delta_{k,r} \Sy_r \Sz_{k+1} \Sx_{r+1}\Big)\\ 
    J_{y} J_{z}\comm{\Sy_{k}\Sy_{k+1}}{\Sz_{r}\Sz_{r+1}} =&\; i J_y J_z \Big( \delta_{k+1,r+1} \Sy_k \Sz_r \Sx_{k+1} + \delta_{k,r+1} \Sz_r \Sx_{k} \Sy_{k+1} \\
    +&\; \delta_{k+1,r} \Sy_k \Sx_{k+1} \Sz_{r+1} +\delta_{k,r} \Sx_k \Sz_{r+1} \Sy_{k+1}\Big) 
\end{align*}
Next step requires us to evaluate the sum over lattice sites to get rid of the Kronecker \(\delta{}\)'s. As before, one of the three parts of the calculations is provided in full detail
\begin{align}
    &i\sum_{r=1}^{L}  J_{x} J_{y}\comm{\Sx_{k}\Sx_{k+1}}{\Sy_{r}\Sy_{r+1}} + i\sum_{r=1}^{L} J_{x} J_{y}\comm{\Sy_{k}\Sy_{k+1}}{\Sx_{r}\Sx_{r+1}} =\nonumber \\
    & - J_x J_y \Big( \cancel{\Sx_k \Sy_k \Sz_{k+1}} +  \Sx_k \Sz_{k+1} \Sy_{k+2} + \Sy_{k-1} \Sz_k \Sx_{k+1} +\bcancel{\Sz_k \Sy_{k+1} \Sx_{k+1}}\Big)\nonumber\\
    &+  J_x J_y \Big(  \cancel{\Sx_k \Sy_k \Sz_{k+1}} + \Sy_k \Sz_{k+1} \Sx_{k+2} + \Sx_{k-1} \Sz_k \Sy_{k+1} + \bcancel{\Sz_k \Sy_{k+1} \Sx_{k+1}}\Big)\nonumber\\
    &= J_x J_y \Big( \Sy_k \Sz_{k+1} \Sx_{k+2} - \Sx_k \Sz_{k+1} \Sy_{k+1} - \left(\Sy_{k-1}\Sz_k \Sx_{k+1} - \Sx_{k-1}\Sz_{k}\Sy_{k+1}\right) \Big)
\end{align}
\begin{align}
    &i\sum_{r=1}^{L}  J_{x} J_{z}\comm{\Sx_{k}\Sx_{k+1}}{\Sz_{r}\Sz_{r+1}} + i\sum_{r=1}^{L} J_{x} J_{z}\comm{\Sz_{k}\Sz_{k+1}}{\Sx_{r}\Sx_{r+1}} = \nonumber\\
    &= J_x J_z \Big( \Sx_k \Sy_{k+1} \Sz_{k+2} - \Sz_k \Sy_{k+1} \Sx_{k+2} - \left(\Sx_{k-1}\Sy_k \Sz_{k+1} - \Sz_{k-1}\Sy_{k}\Sx_{k+1}\right) \Big)
\end{align}
\begin{align}    
    &i\sum_{r=1}^{L}  J_{y} J_{z}\comm{\Sy_{k}\Sy_{k+1}}{\Sz_{r}\Sz_{r+1}} + i\sum_{r=1}^{L} J_{y} J_{z}\comm{\Sz_{k}\Sz_{k+1}}{\Sy_{r}\Sy_{r+1}} = \nonumber\\
    &= J_y J_z \Big( \Sz_k \Sx_{k+1} \Sy_{k+2} - \Sy_k \Sx_{k+1} \Sz_{k+2} - \left(\Sz_{k-1}\Sx_k \Sy_{k+1} - \Sy_{k-1}\Sx_{k}\Sz_{k+1}\right) \Big)
\end{align}
What now remains is to collect these parts and see that we finally arrive at the equation for the energy current density
\begin{align}
    j_k^E &= J_x J_y \left(\Sy_{k-1}\Sz_k \Sx_{k+1} - \Sx_{k-1}\Sz_{k}\Sy_{k+1}\right) \nonumber \\
    &+ J_x J_z \left(\Sx_{k-1}\Sy_k \Sz_{k+1} - \Sz_{k-1}\Sy_{k}\Sx_{k+1}\right) \nonumber \\
    &+ J_y J_z \left(\Sz_{k-1}\Sx_k \Sy_{k+1} - \Sy_{k-1}\Sx_{k}\Sz_{k+1}\right) \nonumber \\
    &= J_x J_y \left(\Sy_{k-1}\Sz_k \Sx_{k+1} - \Sx_{k-1}\Sz_{k}\Sy_{k+1}\right) + \text{cyclic permutations of } \left(x,y,z\right)
    \label{eq:energy current density}
\end{align}
which is precisely the expression from~\textcite{Zotos1997}. 
However, in this work we are interested in the XXZ model with the Hamiltonian~\eqref{eq:XXZ}. To this end,
we need to set \(J_x, J_y = J\), \(J_z = J \Delta \) and substitute \(\Sx_k = \frac{\Sp_k + \Sm_k}{2}\), \(\Sy_k = \frac{\Sp_k-\Sm_k}{2i}\).
After some more lengthy calculations, we finally arrive at the desired form of energy current density operator
\begin{align}
    j_k^E &= i \Bigg( \underbrace{\frac{J^2}{2} \Sm_{k-1}\Sz_k \Sp_{k+1} + \frac{J^2\Delta}{2} \Sz_{k-1}\Sp_k \Sm_{k+1} + \frac{J^2\Delta}{2} \Sp_{k-1}\Sm_k\Sz_{k+1}}_{O_k} \nonumber \\
    &- \underbrace{\left(\frac{J^2}{2} \Sp_{k-1}\Sz_k \Sm_{k+1} + \frac{J^2\Delta}{2} \Sz_{k-1}\Sm_k \Sp_{k+1} + \frac{J^2\Delta}{2} \Sm_{k-1}\Sp_{k} \Sz_{k+1} \right)}_{O_k^{\dagger}}\Bigg) \nonumber \\
    &= i\left(O_k - O_k^{\dagger}\right)
    \label{eq:final energy current density}
\end{align}
It is evident that the energy current operator \(J^E = \sum_{k=1}^L i \left(O_k - O_k^{\dagger}\right) \) has support of exactly \(3\) consecutive sites.

