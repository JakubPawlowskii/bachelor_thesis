\chapter{Decay of integrals of motions in weakly perturbed XXZ model}
\thispagestyle{chapterBeginStyle}

So far we have focused on investigating properties of an \textit{integrable} XXZ
spin-\(1/2\) chain. 

\textcolor{blue}{Write, this part after the introduction to avoid repetition.}


\section{Adding perturbation to the Hamiltonian}
We are now going to weakly break integrability by adding a suitable perturbation.
New Hamiltonian has the following form:
\begin{equation}
    H_{XXZ} = \frac{1}{2}\sum_{j = 1}^{L}\left( S^{+}_{j} S^{-}_{j+1} + 
    S^{-}_{j}S^{+}_{j+1} \right) + \Delta\sum_{j = 1}^{L} S^{z}_{j}S^{z}_{j+1}
    + \alpha H'
    \label{eq:HXXZ perturbed}
\end{equation}
where \(H'\) is the perturbation that breaks integrability for nonzero \(\alpha \):
\begin{equation}
    H'=\sum_{j = 1}^{L} S^{z}_{j}S^{z}_{j+2}
    \label{eq:perturbation}
\end{equation}
In such system, only two conserved quantities remain --- the Hamiltonian \(H_{XXZ}\) itself 
and the total magnetization \(\Sz_{tot}\). All other integral of motions cease to be conserved
and decay with a finite relaxation time \(\tau\). We are interested in investigating this
decay and the timescales involved. However at first we should establish a range of
values of parameter \(\alpha\) so its small enough that \(H'\) remains a perturbation
but large enough to be relevant for finite system sizes accessible numerically. 
To this end we take the previously discussed (Q)LIOMs  \(J^E, \hat{O}_1,\hat{O}_2\)
and investigate their behavior under finite-time averaging (as defined in~\eqref{def:simple time avg})
generated by perturbed Hamiltonian. More precisely, we calculate the finite-time autocorrelation
function (finite-time stiffness) \(\lambda^{\tau}_A = \hs{\bar{A}^{\tau}}{\bar{A}^{\tau}}\).
In order to relate this to the discussion of spectral functions in Section~\ref{sec:spectral function},
we will from now on use the cutoff frequency \(\omega = \frac{1}{\tau}\), instead of the time of
averaging \(\tau\).









\section{Relaxation of known (Q)LIOMs}



  For example:
  \begin{align*}
    &\hs{Q_{\alpha}(t)}{Q_{\alpha}} = e^{-\frac{t}{\tau_{\alpha}}} \implies 
    S(\omega) = \sum_{\alpha} D_{\alpha} \frac{1}{\pi}\frac{\tau_{\alpha}}{{\left(\omega \tau_{\alpha}\right)}^2 + 1}\\
    &\implies I(\omega) = \sum_{\alpha}D_{\alpha}\frac{2}{\pi} \arctan{(\tau_{\alpha}  \omega )}
  \end{align*}
To investigate the decay of LCQs let us define the following:
\begin{equation*}
  R(\omega,\alpha) = \frac{I(\omega,\alpha)}{\lim_{\omega \to 0^{+}} I(\omega,\alpha=0)} 
\end{equation*}
where:
\begin{align*}
  &I(\omega) = \frac{1}{\mathcal{D}}\sum_{n,m=1}^{\mathcal{D}}\theta\left(\omega -\abs{E_n-E_m}\right) \abs{\matrixel{m}{\hat{A}}{n}}^2\\
  &\lim_{\omega \to 0^{+}} I(\omega) = \sum_{\substack{n,m=1 \\ E_n=E_m}}^{\mathcal{D}} \abs{\matrixel{m}{\hat{A}}{n}}^2
\end{align*}  
