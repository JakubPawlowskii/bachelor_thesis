\chapter{Energy current}
\thispagestyle{chapterBeginStyle}
\label{chap:energy current}

\paragraph{}In order to test our QLIOM finding algorithm and the correctness of its implementation, we investigate the known case of
energy current in spin\(-1/2\) XXZ model~\autocite*{Mierzejewski2015Approx}. For the sake of completeness, derivation of
spin energy current for the general XYZ model will be presented, following the definitions in \autocite{Zotos1997}.
We start with the XYZ Hamiltonian with periodic boundary conditions:
\begin{equation}
    H_{XYZ} = \sum_{i=1}^L  \left( J_{x} \Sx_{i}\Sx_{i+1} + J_{x} \Sy_{i}\Sy_{i+1} + J_{z} \Sz_{i}\Sz_{i+1} \right)
\end{equation}
It is easy to see that this Hamiltonian can be represented as a sum of operators supported on two consecutive sites:
\begin{equation}
    H_{XYZ} = \sum_{i=1}^L h_{i,i+1}
\end{equation}
where \(h_{i,i+1} = J_{x} \Sx_{i}\Sx_{i+1} + J_{x} \Sy_{i}\Sy_{i+1} + J_{z} \Sz_{i}\Sz_{i+1} \) and periodic boundary conditions
require that \(h_{L,L+1} = h_{L,1}\). The energy operator is a conserved quantity, thus the time evolution of its local density
is given by the discrete continuity equation:
\begin{equation}
    \dv{h_{i,i+1}(t)}{t} + \div{j_{i}^{E}} = 0 
    \label{eq:discrete continuity}
\end{equation}
where \(\div{j_{i}^E} \equiv j_{i+1}^E - j_{i}^E\). On the other hand, time evolution of an arbitrary operator is determined
by the Heisenberg equations:
\begin{equation}
    \dv{h_{i,i+1}(t)}{t} = i \comm{H_{XYZ}}{h_{i,i+1}(t)}
    \label{eq:heisenberg equation}
\end{equation}
Combining equations~\ref{eq:discrete continuity} and~\eqref{eq:heisenberg equation} we obtain the defining equations for
the spin energy current:
\begin{equation}
    j_{i+1}^E - j_{i}^E = - i \comm{H_{XYZ}}{h_{i,i+1}(t)} = i \comm{h_{i,i+1}(t)}{H_{XYZ}} = i \sum_{k=1}^L \comm{h_{i,i+1}}{h_{k,k+1}} 
    \label{eq:energy current defining equation}
\end{equation}
Similar equations can be written for any operator being a sum of local operators (\textcolor{blue}{with support 2?}) such as
the total spin operator or particle number operator in Hubbard model. Equation~\eqref{eq:energy current defining equation} is
conceptually simple, yet quite tedious to solve due to the amount of commutators present. Luckily, leveraging commutator properties
to our advantage will allow us to simplify the calculations. Let us begin with inserting the definition of \(h_{i,i+1}\) into 
equation~\eqref{eq:energy current defining equation}:
\begin{align*}
    \comm{h_{i,i+1}}{h_{k,k+1}} =& \comm{J_{x} \Sx_{i}\Sx_{i+1} + J_{x} \Sy_{i}\Sy_{i+1} + J_{z} \Sz_{i}\Sz_{i+1}}{J_{x} \Sx_{k}\Sx_{k+1} + J_{x} \Sy_{k}\Sy_{k+1} + J_{z} \Sz_{k}\Sz_{k+1}}\\
    =& J_{x} J_{y}\comm{\Sx_{i}\Sx_{i+1}}{\Sy_{k}\Sy_{k+1}} + J_{x} J_{z}\comm{\Sx_{i}\Sx_{i+1}}{\Sz_{k}\Sz_{k+1}} + J_{y} J_{x}\comm{\Sy_{i}\Sy_{i+1}}{\Sx_{k}\Sx_{k+1}}\\
    +& J_{y} J_{z}\comm{\Sy_{i}\Sy_{i+1}}{\Sz_{k}\Sz_{k+1}} + J_{z} J_{x}\comm{\Sz_{i}\Sz_{i+1}}{\Sx_{k}\Sx_{k+1}} + J_{z} J_{y}\comm{\Sz_{i}\Sz_{i+1}}{\Sy_{k}\Sy_{k+1}}  
\end{align*}
By inspection it becomes clear that out of six terms present, only three will need to be directly evaluated, as commutators of the form
\(xy\) will differ from \(yx\) by a sign and an index change.

\begin{align*}
    J_{x} J_{y}\comm{\Sx_{i}\Sx_{i+1}}{\Sy_{k}\Sy_{k+1}} =& J_x J_y \Big(\Sx_i\comm{\Sx_{i+1}}{\Sy_{k}\Sy_{k+1}} + \comm{\Sx_{i}}{\Sy_{k}\Sy_{k+1}} \Sx_{i+1} \Big)\\
    =& J_x J_y \Big( \Sx_i \left( \Sy_k \comm{\Sx_{i+1}}{\Sy_{k+1}} + \comm{\Sx_{i+1}}{\Sy_{k}} \Sy_{k+1}\right) + 
    \left( \Sy_k \comm{\Sx_{i}}{\Sy_{k+1}} + \comm{\Sx_{i}}{\Sy_{k}} \Sy_{k+1} \Sy_{k+1}\right) \Big)
\end{align*}