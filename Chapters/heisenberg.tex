\chapter{XXZ model\label{sec:xxz}}
\thispagestyle{chapterBeginStyle}

We investigate a one dimensional XXZ Hamiltonian on a one-dimensional lattice of \(L\) sites with periodic boundary conditions.
Throughout this thesis we will work in units such that \(\hbar = 1\).

Spin operator algebra:
\begin{align*}
    &\comm{S_{i}^{\alpha}}{S_{k}^{\beta}} = i  \delta_{i,k}\epsilon_{\alpha \beta \gamma} S_{i}^{\gamma}\\
    &S_{i}^{\pm} = \Sx_i \pm i \Sy_i\\
    &\comm{\Sp_i}{\Sp_k} = 2 \delta_{i,k} \Sz_i \\
    &\comm{\Sz_i}{S^{\pm}_k} = \pm \delta_{i,j} S_{i}^{\pm}
\end{align*}

\textcolor{blue}{Write about tensor product, Hilbert space structure and such}

\noindent Heisenberg Hamiltonian:
\begin{equation}
    H_{XXZ} = J\sum_{j = 1}^{L}\left( S^{+}_{j} S^{-}_{j+1} + S^{-}_{j}S^{+}_{j+1} \right) + J\Delta\sum_{j = 1}^{L} S^{z}_{j}S^{z}_{j+1}
    + \alpha H'
    \label{eq:HXXZ}
\end{equation}
where \(H'\) is the perturbation that breaks integrability for nonzero \(\alpha\):
\begin{equation}
    H'=\sum_{j = 1}^{L} S^{z}_{j}S^{z}_{j+2}
\end{equation}
We set \(J = 1\) and subsequently work in units of \(J\).

It is perhaps important to note that, between spin operators in the Hamiltonian stands
the tensor product, not regular composition of operators.