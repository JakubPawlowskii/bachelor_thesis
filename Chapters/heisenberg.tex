\chapter{XXZ model\label{sec:xxz}}
\thispagestyle{chapterBeginStyle}
    It is well known, that we can divide magnetic materials into two broad groups: those which
exhibit magnetic properties in reaction to external magnetic field and those which have a nonzero
magnetic moment without external field~\autocite{spalek2015}. First group consists of paramagnetic
and diamagnetic systems. In the former, nonzero net magnetic moment comes from alignment of
valence electrons' spins in the direction of external magnetic field. In the latter, we deal with
 an inductive effect in which external field induces magnetic dipoles opposing the field tha have
induced them. Diamagnetism exists in all materials, however it is usually much weaker than other magnetism related
effects and thus only detectable in the absence of them. Second group includes ferromagnets, which
exhibit spontaneous magnetization below Curie temperature, and ferrimagnets, which are 
composed of two ferromagnetic sublattices with different spontaneous magnetization. There
 are also antiferromagnets, which are essentially a special case of ferrimagnets in which the two sublattices
, below the so-called N{\'e}el temperature, have spontaneous magnetizations of equal magnitude
but opposite directions~\autocite{nolting2018theoretical}.

There are two paradigmatic models of magnetism, namely the Heisenberg model, which describes magnetism
of localized electrons and their magnetic moments (spins), and the Hubbard model which deals with
magnetism of delocalized electrons, called the itinerant magnetism. In this thesis we will focus on
a special case of the former of two models, namely the XXZ model.


\section{Heisenberg-Dirac exchange interaction}
We will now proceed with a derivation and a physical motivation of Heisenberg model.
Our discussion will be based on the books by~\textcite{spalek2015} and 
thesis by~\textcite{Ng2011HeisenbergM}.
The story begins with two electrons interacting with each other via Coulomb potential.
An electron can be described by two quantities, its position in space and its spin.
Two facilitate these two degrees of freedom, we say that \(i\)-th electron's wavefunction
lives in a Hilbert space which is a tensor product of spacial 
wavefunction space \(\mathcal{H}_i \cong L^2(\RR^3) \otimes \mathfrak{h}_i \), where
\(L^2(\RR^3)\) is the usual space of square-integrable functions on \(\RR^3\),
and spin wavefunctions space \(\mathfrak{h} \cong \CC^2\) is a two-dimensional vector space spanned by
\(\ket{\uparrow} = \binom{1}{0} \) and \(\ket{\downarrow} = \binom{0}{1}\).
The combined wavefunction of a composite two-particle system it then an element of
\(\mathcal{H}_1 \otimes \mathcal{H}_2\), which can be decomposed into spacial and spin
components, i.e\ \(\mathcal{H}_1 \otimes \mathcal{H}_2 \cong\mathcal{H}_{\mathrm{space}} 
\otimes \mathcal{H}_{\mathrm{spin}}\), where \(\mathcal{H}_{\mathrm{space}} 
\cong L^2(\RR^3) \otimes L^2(\RR^3)\) and \(\mathcal{H}_{\mathrm{spin}} \cong
\mathfrak{h}_1 \otimes \mathfrak{h}_2\). 

Hamiltonian of two interacting electrons is given by:
\begin{equation}
    H_C = \underbrace{-\frac{\hbar^2}{2m} \laplacian_1  
    -\frac{\hbar^2}{2m} \laplacian_2 }_{\textrm{free particles}}
     + \underbrace{V(\bm{r}_1,\bm{r}_2)}_{\textrm{interaction}}
     \label{eq:Coulomb Hamiltonian}
\end{equation}
where in case of Coulomb interaction we have \(V(\bm{r}_1,\bm{r}_2)=\frac{e^2}
{\abs{\bm{r}_1-\bm{r}_2}}\).
Formally, this Hamiltonian acts on the space \(\mathcal{H}_1 \otimes \mathcal{H}_2\).
However, it depends only on the spatial coordinates \(\bm{r}_1,\bm{r}_2\) and not on
the spin coordinates, so essentially
its actions is restricted to the \(\mathcal{H}_{\textrm{space}}\) part of the full Hilbert space.
This is a crucial observation that lead to the development of Heisenberg model. We will now
seek a way to replace this Hamiltonian by an equivalent one acting only on 
\(\mathcal{H}_{\textrm{spin}}\).

It is time to invoke the Pauli exclusion principle, which requires the composite wavefunction
of two electrons to be antisymmetric under exchange of pairs of coordinates (both spatial and
spin degrees of freedoms are treated like coordinates). Because 
Hamiltonian~\ref{eq:Coulomb Hamiltonian} does not depend explicitly on spin, the total
wavefunction \(\psi\) can be expressed as tensor product \(\psi_{\mathrm{space}} 
\otimes \psi_{\mathrm{spin}} \). Antisymmetric nature of \(\psi\) then requires one 
of these components to be antisymmetric \((a)\) and the other to be symmetric \((s)\). 
Spatial wavefunctions are of the form:
\begin{align}
    &\psi_{\textrm{space}}^{(s)} = \psi_1(\bm{r}_1)\otimes \psi_2(\bm{r}_2) +
    \psi_2(\bm{r}_1)\otimes \psi_1(\bm{r}_2) \\
    &\psi_{\textrm{space}}^{(a)} = \psi_1(\bm{r}_1)\otimes \psi_2(\bm{r}_2) -
    \psi_2(\bm{r}_1)\otimes \psi_1(\bm{r}_2)
\end{align}
where \(\psi_1,\psi_2 \in L^2(\RR^3)\). Spin wavefunctions are elements of \(\CC^2\)
and are given by:
\begin{align}
    &\psi_{\textrm{spin}}^{(s)} = \ket{\uparrow\uparrow},\; 
    \ket{\uparrow\downarrow} + \ket{\downarrow\uparrow},\; \ket{\downarrow\downarrow} \\
    &\psi_{\textrm{spin}}^{(a)} = \ket{\uparrow\downarrow}-\ket{\downarrow\uparrow}
\end{align}
where \(\ket{\uparrow\uparrow}\) is an usual shorthand notation for \(\ket{\uparrow}_1 \otimes
\ket{\uparrow}_2\). Moreover, symmetric spin wavefunctions
constitute a triplet state, whereas antisymmetric one is a singlet state. 

Possible two-electron wavefunctions are thus either \(\varphi=\psi_{\textrm{space}}^{(s)}
\otimes \psi_{\textrm{spin}}^{(a)}\) or \(\chi = \psi_{\textrm{space}}^{(a)}
\otimes \psi_{\textrm{spin}}^{(s)}\). Expected value of energy of Coulomb interaction 
in these states is given by:
\begin{align}
    &\expectationvalue{H_C}{\varphi} = \expectationvalue{V}{\psi_{\textrm{space}}^{(s)}} = E^{(s)}\\
    &\expectationvalue{H_C}{\chi} = \expectationvalue{V}{\psi_{\textrm{space}}^{(a)}} = E^{(a)} 
\end{align}
Because \(\psi_{\textrm{space}}^{(a)}\) is symmetric with respect to \((\bm{r}_1-\bm{r}_2)\)
we have \(E^{(s)}>E^{(a)}\). Therefore, it is energetically
favourable for our system to pick the total wavefunction that is antisymmetric
in space and symmetric in spin coordinates.

Under the Coulomb interaction, symmetric and antisymmetric spin wavefunctions are not
directly distinguished. It is the difference between spatial parts, together with Pauli
exclusion principle that forces the choice of a triplet state. Let us now do something
similar, but the other way around. We can formally recast the Coulomb Hamiltonian as a
spin-spin interaction acting on \(\mathcal{H}_{\textrm{spin}}\) that would distinguish
between symmetric and antisymmetric spin wavefunctions and thus fix the spatial part.
Let \(\tau^x,\tau^y,\tau^z\) be the \(2\cross 2\) Pauli matrices:
\begin{equation}
\tau^x = \mqty(\pmat{1}),\;\;\;\tau^y = \mqty(\pmat{2}),\;\;\;\tau^z= \mqty(\pmat{3})
\label{eq:Pauli matrices}
\end{equation}
Together with a \(2\cross 2\) identity matrix they form a basis of vector space of
Hermitian operators acting on a single spin Hilbert space.
They are traceless and of unit determinant. By direct computation it can be checked that
they satisfy a particular commutation and anticommutation relations:
\begin{align}
    &\comm{\tau^j}{\tau^k} = 2 i \varepsilon_{jkl}\tau^{l}\\
    &\anticommutator{\tau^j}{\tau^k} = 2 \delta_{jk} \Id_{2\cross2}
\end{align}
which in turn leads to the following important property:
\begin{align}
        \comm{\tau^{j}}{\tau^{k}} + \anticommutator{\tau^{j}}{\tau^{k}} 
    &=\left(\tau^{j} \tau^{k}-\tau^{k} \tau^{j}\right)+\left(\tau^{j} \tau^{k}+\tau^{k} 
    \tau^{j}\right) \nonumber \\
    i \varepsilon_{j k l} \tau^{l}+\delta^{j k} I &= \tau^{j} \tau^{k}
\end{align}
We define an operator:
\begin{equation}
    \bm{\tau}_1 \cdot \bm{\tau}_2 = \tau_1^x \otimes \tau_2^x + \tau_1^y \otimes \tau_2^y +
    \tau_1^z \otimes \tau_2^z  
\end{equation}
where subscripts refer to which electron's Hilbert space they act on. 
Let us now examine how this operator acts on \(\ket{\uparrow\uparrow}\) spin wavefunction:
\begin{align*}
    \bm{\tau}_1 \cdot \bm{\tau}_2 \ket{\uparrow\uparrow} &= 
    \left(\tau_1^x \ket{\uparrow}_1\right) \otimes \left(\tau^x_2 \ket{\uparrow}_2\right) + 
    \left(\tau_1^y \ket{\uparrow}_1\right) \otimes \left(\tau^y_2 \ket{\uparrow}_2\right) +
    \left(\tau_1^z \ket{\uparrow}_1\right) \otimes \left(\tau^z_2 \ket{\uparrow}_2\right)\nonumber\\
    &= \left( \smqty(\pmat{1}) \smqty(1\\0)\right) \otimes \left( \smqty(\pmat{1}) \smqty(1\\0)\right) +
    \left( \smqty(\pmat{2}) \smqty(1\\0)\right) \otimes \left( \smqty(\pmat{2}) \smqty(1\\0)\right)+
    \left( \smqty(\pmat{3}) \smqty(1\\0)\right) \otimes \left( \smqty(\pmat{3}) \smqty(1\\0)\right)\\
    &= \smqty(0\\1) \otimes \smqty(0\\1) + \smqty(0\\i) \otimes \smqty(0\\i) + 
    \smqty(1\\0) \otimes \smqty(1\\0) = \ket{\uparrow\uparrow}
\end{align*}
So \(\ket{\uparrow\uparrow}\) is an eigenvector of \(\bm{\tau}_1 \cdot \bm{\tau}_2\) with
an eigenvalue \(1\). Carrying out such computations for the three remaining states we get
that all symmetric states are eigenvectors with eigenvalue \(1\), whereas the
antisymmetric state is also an eigenvector, but with eigenvalue \(-3\). 
We could have also obtained this result in a more elegant way by referring directly
to quantum mechanics and algebra of spin angular momentum~\autocite{Sakurai2017}.
If we set \(\hbar = 1\) (which from now on will always be the case), we have the usual
spin vector operators \(\bm{S}_i = \left(\Sx_i,\Sy_i, \Sz_i\right) \) where
 \(S_i^{\alpha} = \tau_i^{\alpha}/2\).
Squares of these operators commute with all other \(S_i^{\alpha}\), hence are the
Casimir operators of their algebras and by Schur's Lemma are proportional to
the identity~\autocite{Woit2017}.
The proportionality constant is equal for spin\(-s\) particles to \(s(s+1)\) and thus
all single spin states are eigenvectors of these operators with eigenvalue \(3/4\). 
We can also construct square of total spin angular momentum operator
\(\bm{S}^2 = (\bm{S}_1 + \bm{2}_2)^2\) for which our triplet (total spin \(S=1\)) and singlet
(total spin \(S=0\)) are eigenvectors with eigenvalue \(S(S+1)\). On the other hand, we 
can calculate \(\bm{S}^2\) directly to obtain the following equation:
\begin{equation}
    S(S+1) = \bm{S}_1^2 + \bm{S}_2^2 + 2\bm{S}_1\cdot \bm{S}_1 = \frac{3}{2} + 2\bm{S}_1\cdot\bm{S}_2
\end{equation} 
Rearranging it, replacing \(\bm{S}_i\) by \(\bm{\tau}_i/2\) and inserting appropriate values
of \(S\) we recreate the previously obtained result on eigenvalues of 
\(\bm{\tau}_1\cdot\bm{\tau}_2\).

After this detour into the world of quantum mechanics of spin, let us return to the problem
at hand. We have established that triplet states and singlet state are eigenvectors of 
\(\bm{\tau}_1 \cdot \bm{\tau}_2\) operators with eigenvalues \(1\) and \(-3\) respectively.
Consider now the following Hamiltonian acting on \(\mathcal{H}_{\textrm{spin}}\):
\begin{equation}
    H_S = \frac{3 E^{(s)} + E^{(a)}}{4} + \frac{E^{(s)} - E^{(a)}}{4} \bm{\tau}_1 \cdot \bm{\tau}_2
    \label{eq:Coulomb recasted}
\end{equation}
