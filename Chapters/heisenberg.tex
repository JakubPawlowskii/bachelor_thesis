\chapter{XXZ model\label{sec:xxz}}
\thispagestyle{chapterBeginStyle}
    It is well known, that we can divide magnetic materials into two broad groups: those which
exhibit magnetic properties in reaction to external magnetic field and those which have a nonzero
magnetic moment without external field~\autocite{spalek2015}. First group consists of paramagnetic
and diamagnetic systems. In the former, nonzero net magnetic moment comes from alignment of
valence electrons' spins in the direction of external magnetic field. In the latter, we deal with
 an inductive effect in which external field induces magnetic dipoles opposing the field tha have
induced them. Diamagnetism exists in all materials, however it is usually much weaker than other magnetism related
effects and thus only detectable in the absence of them. Second group includes ferromagnets, which
exhibit spontaneous magnetization below Curie temperature, and ferrimagnets, which are 
composed of two ferromagnetic sublattices with different spontaneous magnetization. There
 are also antiferromagnets, which are essentially a special case of ferrimagnets in which the two sublattices
, below the so-called N{\'e}el temperature, have spontaneous magnetizations of equal magnitude
but opposite directions~\autocite{nolting2018theoretical}.

There are two paradigmatic models of magnetism, namely the Heisenberg model, which describes magnetism
of localized electrons and their magnetic moments (spins), and the Hubbard model which deals with
magnetism of delocalized electrons, called the itinerant magnetism. In this thesis we will focus on
a special case of the former of two models, namely the XXZ model.


\section{Heisenberg model of magnetism}
We will now proceed with a derivation and a physical motivation of Heisenberg model,
which is Dirac exchange interaction.
Our discussion will be based on the books by~\textcite{spalek2015,Korepin1993} and 
thesis by~\textcite{Ng2011HeisenbergM}.
The story begins with two electrons interacting with each other via Coulomb potential.
An electron can be described by two quantities, its position in space and its spin.
Two facilitate these two degrees of freedom, we say that \(i\)-th electron's wavefunction
lives in a Hilbert space which is a tensor product of spacial an spin 
wavefunction spaces \(\mathcal{H}_i \cong L^2(\RR^3) \otimes \mathfrak{h}_i \), where
\(L^2(\RR^3)\) is the usual space of square-integrable functions on three-dimensional space
and \(\mathfrak{h} \cong \CC^2\) is a two-dimensional vector space spanned by
\(\ket{\uparrow} = \binom{1}{0} \) and \(\ket{\downarrow} = \binom{0}{1}\).
The combined wavefunction of a composite two-particle system it then an element of
\(\mathcal{H}_1 \otimes \mathcal{H}_2\), which can be decomposed into spacial and spin
components, i.e\ \(\mathcal{H}_1 \otimes \mathcal{H}_2 \cong\mathcal{H}_{\mathrm{space}} 
\otimes \mathcal{H}_{\mathrm{spin}}\), where \(\mathcal{H}_{\mathrm{space}} 
\cong L^2(\RR^3) \otimes L^2(\RR^3)\) and \(\mathcal{H}_{\mathrm{spin}} \cong
\mathfrak{h}_1 \otimes \mathfrak{h}_2\). This decomposition is possible because of 
a canonical isomorphism \(V_1 \otimes V_2 \cong V_2 \otimes V_1\). 

Hamiltonian of two interacting electrons is given by:
\begin{equation}
    H = \underbrace{-\frac{\hbar^2}{2m} \laplacian_1 + V_1(\bm{r}_1) + 
    -\frac{\hbar^2}{2m} \laplacian_2 + V_2(\bm{r}_2)}_{\textrm{single particle}}
     + \underbrace{V(\bm{r}_1,\bm{r}_2)}_{\textrm{interaction}}
\end{equation}
where in case of Coulomb interaction we have \(V(\bm{r}_1,\bm{r}_2)=\frac{e^2}
{\abs{\bm{r}_1-\bm{r}_2}}\), but this exact expression in not important in this reasoning.








% We investigate a one dimensional XXZ Hamiltonian on a one-dimensional lattice of \(L\) sites with periodic boundary conditions.
% Throughout this thesis we will work in units such that \(\hbar = 1\).

% Spin operator algebra:
% \begin{align*}
%     &\comm{S_{i}^{\alpha}}{S_{k}^{\beta}} = i  \delta_{i,k}\epsilon_{\alpha \beta \gamma} S_{i}^{\gamma}\\
%     &S_{i}^{\pm} = \Sx_i \pm i \Sy_i\\
%     &\comm{\Sp_i}{\Sp_k} = 2 \delta_{i,k} \Sz_i \\
%     &\comm{\Sz_i}{S^{\pm}_k} = \pm \delta_{i,j} S_{i}^{\pm}
% \end{align*}

% \textcolor{blue}{Write about tensor product, Hilbert space structure and such}

% \noindent Heisenberg Hamiltonian:
% \begin{equation}
%     H_{XXZ} = \frac{J}{2}\sum_{j = 1}^{L}\left( S^{+}_{j} S^{-}_{j+1} + S^{-}_{j}S^{+}_{j+1} \right) + J\Delta\sum_{j = 1}^{L} S^{z}_{j}S^{z}_{j+1}
%     + \alpha H'
%     \label{eq:HXXZ}
% \end{equation}
% where \(H'\) is the perturbation that breaks integrability for nonzero \(\alpha \):
% \begin{equation}
%     H'=\sum_{j = 1}^{L} S^{z}_{j}S^{z}_{j+2}
% \end{equation}
% We set \(J = 1\) and subsequently work in units of \(J\).

% It is perhaps important to note that, between spin operators in the Hamiltonian stands
% the tensor product, not regular composition of operators.
