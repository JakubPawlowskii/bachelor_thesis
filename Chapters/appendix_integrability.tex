\chapter{Classical and quantum integrability\label{app:int}}
\thispagestyle{chapterBeginStyle}

\paragraph{Classical integrability}Before jumping head first into the quantum realm, it is perhaps valuable the state
the problem of classical integrability and results therein.
We will consider this topic in the spirit of classical discussion
of Liouville and Arnold as presented in a monograph by~\textcite{Gutzwiller1991}.

One usually begins the discussion of classical mechanics with the Lagrangian picture and then
via Legendre transform moves on to the Hamiltonian picture, however we shall skip this
for brevity and take the Hamiltonian picture as a starting point.
A dynamical system at every given time \(t\) is described by its momentum \(p\) and
position \(q\). If we assume our system to have \(n\) degrees of freedoms, then
its state can be specified as a point in a \(2n\)-dimensional space (\(n\) for momenta and
\(n\) for coordinates) called the phase space. Dynamics are governed by a function
of \(H = H(p,q,t)\) called the Hamiltonian, which can be interpreted as the energy,
and the so-called \textit{Hamitlon-Jacobi equations of motion}
\begin{equation}
    \dv{p_j}{t} = -\pdv{H}{q_j},\;\;\;\; \dv{q_j}{t} = \pdv{H}{p_j}
    \label{eq:Hamilton equations}
\end{equation} 
Suppose that we have another function on a phase space, say \(F=F(p,q)\), with
a property that its value does not vary in time if we take as \(p\) and \(q\) solutions
of equations~\eqref{eq:Hamilton equations}. This is equivalent to the following condition
\begin{align}
    0 = \dv{}{t}F(p,q) &= \pdv{F}{p} \dv{p}{t} + \pdv{F}{q}\dv{q}{t} \nonumber\\
    &= \pdv{H}{p}\dv{F}{q} - \pdv{H}{q}\dv{F}{p} = \poissonbracket{H}{F}
\end{align}
where \(\poissonbracket{\bullet }{\bullet }\) is the Poisson bracket.
Vanishing of Poisson bracket with Hamiltonian is a necessary condition for \(F\)
to be called an \textit{integral of motion}. This property has a nice geometric interpretation.
On the one hand, the vector field \(\left(-\pdv*{H}{q},\pdv*{H}{p}\right)\) in a phase space is
tangent to surface \(F(p,q) = const\). On the other hand, the vector field 
\(\left(-\pdv*{F}{q},\pdv*{F}{p}\right)\) is tangent to the energy surface \(H(p,q) = E\).
The trajectory of a dynamical system in a phase space is then located in the intersection
of these two surfaces.
Let us now imagine that we have a set \(\set{F_i}_{i=1,\ldots,k}\) of such conserved
quantities that are functionally independent (no function from this set can be expressed
as a function of other functions from this set). It is not yet sufficient to enable us to solve
the equations of motion. The integrals of motion must also be in \textit{involution},
which means that for any \(F_i,F_j\in\set{F_i}_{i=1,\ldots,k}\) we have
\(\poissonbracket{F_i}{F_j} = 0\). Existence of \(k\) integrals of motion restricts
the dynamics of a system to a \((2n-k)\)-dimensional subspace of the phase space, whereas the fact
that they are in involution guarantees that this subspace has a simple internal structure.
If there are at least \(n\) such conserved quantities, then a famous theorem by
Liouville holds, which states a system with \(n\) degrees of freedom with
\(n\) integrals of motion is integrable by quadratures.
Then, there exists such a canonical transformation \((p,q)\to (F,\Theta)\), to the
so-called angle-action variables, that \(H(p,q) = H(F)\) and the equations
of motions are solved by \(F_j(t) = F_j^0\) and \(\Theta(t)_j = \Omega_j t + \Theta_j(0)\)~\autocite{arnold2013mathematical}.
Moreover, it was proven with the help of topology that this restricted subspace
has a shape of an \(n\)-dimensional torus, called the \textit{invariant torus}. In
general different initial conditions correspond to different invariant tori.
So we see that integrability in classical mechanics has a very precise 
definition --- differential equations governing time evolution can be solved explicitly
with the help of action-angle variables. Then, the solutions to equations~\ref{eq:Hamilton equations}
for integrable of motions exhibit periodic dynamics constrained to some invariant torus.
On the other hand, solutions for nonintegrable systems, given sufficient time, explore the whole
phase space.

\paragraph{KAM theorem}Before we leave the classical world, let us ponder upon one more question.
What happens to a classical integrable system under small perturbation? More precisely,
how does the breakdown of invariant tori looks like? The answer to this question lies
within the remarkable KAM theorem, which was proven by a joint effort
of~\textcite{Kolmogorov1954,Moser1962,Arnold1963}. For
a system with finite number of degrees of freedom it specifies, under some assumptions,
that majority of the invariant tori that occupy\footnote{The more technical term is
\textit{foliate}.} the phase space survive the influence of small perturbations. This entails
the possibility of coexistence between chaos and regularity. Eventually, as perturbation grows,
chaotic regions fill the phase space completely~\autocite{DAlessio2016}. We will end this part
with a quote from Arnold's book~\autocite{arnold2013mathematical} about the KAM theorem.
\begin{quotation}
   \textit{ If an unperturbed system is nondegenerate, then for sufficiently
small conservative hamiltonian perturbations, most non-resonant invariant
tori do not vanish, but are only slightly deformed, so that in the phase space
of the perturbed system, too, there are invariant tori densely filled with phase
curves winding around them conditionally-periodically, with a number of
independent frequencies equal to the number of degrees of freedom.
These invariant tori form a majority in the sense that the measure of the
complement of their union is small when the perturbation is small.}
\end{quotation}
\paragraph{Quantum integrability} No notion of a phase space.

